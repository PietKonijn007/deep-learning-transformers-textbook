\chapter{Calculus and Optimization}
\label{chap:calculus_optimization}

\section*{Chapter Overview}

Training deep learning models requires optimizing complex, high-dimensional functions. This chapter develops the calculus and optimization theory necessary to understand how neural networks learn from data. We cover multivariable calculus, gradient computation, and the optimization algorithms that power modern deep learning.

The centerpiece of this chapter is backpropagation, the algorithm that efficiently computes gradients in neural networks. We derive backpropagation from first principles, showing how the chain rule enables gradient computation through arbitrarily deep computational graphs. We then explore gradient descent and its variants, which use these gradients to iteratively improve model parameters.

\subsection*{Learning Objectives}

After completing this chapter, you will be able to:

\begin{enumerate}
    \item Compute gradients and Jacobians for multivariable functions
    \item Apply the chain rule to composite functions
    \item Understand and implement the backpropagation algorithm
    \item Implement gradient descent and its variants (SGD, momentum, Adam)
    \item Analyze convergence properties of optimization algorithms
    \item Apply learning rate schedules and regularization techniques
\end{enumerate}

\section{Multivariable Calculus}
\label{sec:multivariable_calculus}

\subsection{Partial Derivatives}

\begin{definition}[Partial Derivative]
\label{def:partial_derivative}
For function $f: \R^n \to \R$, the \textbf{partial derivative} with respect to $x_i$ is:
\begin{equation}
\frac{\partial f}{\partial x_i} = \lim_{h \to 0} \frac{f(x_1, \ldots, x_i + h, \ldots, x_n) - f(x_1, \ldots, x_i, \ldots, x_n)}{h}
\end{equation}
\end{definition}

\begin{example}[Computing Partial Derivatives]
\label{ex:partial_derivatives}
For $f(x_1, x_2) = x_1^2 + 3x_1 x_2 + x_2^2$:
\begin{align}
\frac{\partial f}{\partial x_1} &= 2x_1 + 3x_2 \\
\frac{\partial f}{\partial x_2} &= 3x_1 + 2x_2
\end{align}

At point $(x_1, x_2) = (1, 2)$:
\begin{align}
\frac{\partial f}{\partial x_1}\bigg|_{(1,2)} &= 2(1) + 3(2) = 8 \\
\frac{\partial f}{\partial x_2}\bigg|_{(1,2)} &= 3(1) + 2(2) = 7
\end{align}
\end{example}

\subsection{Gradients}

\begin{definition}[Gradient]
\label{def:gradient}
For function $f: \R^n \to \R$, the \textbf{gradient} is the vector of partial derivatives:
\begin{equation}
\nabla f(\vx) = \begin{bmatrix} 
\frac{\partial f}{\partial x_1} \\
\frac{\partial f}{\partial x_2} \\
\vdots \\
\frac{\partial f}{\partial x_n}
\end{bmatrix} \in \R^n
\end{equation}
\end{definition}

The gradient points in the direction of steepest ascent of the function.

\begin{example}[Gradient of Loss Function]
\label{ex:loss_gradient}
For mean squared error loss:
\begin{equation}
L(\vw) = \frac{1}{N} \sum_{i=1}^N (y_i - \vw\transpose \vx^{(i)})^2
\end{equation}

The gradient with respect to $\vw$ is:
\begin{equation}
\nabla_{\vw} L = -\frac{2}{N} \sum_{i=1}^N (y_i - \vw\transpose \vx^{(i)}) \vx^{(i)}
\end{equation}

For $N=1$, $\vw = [w_1, w_2]\transpose$, $\vx = [1, 2]\transpose$, $y = 5$, and current prediction $\hat{y} = \vw\transpose \vx = 3$:
\begin{equation}
\nabla_{\vw} L = -2(5 - 3) \begin{bmatrix} 1 \\ 2 \end{bmatrix} = \begin{bmatrix} -4 \\ -8 \end{bmatrix}
\end{equation}

The negative gradient $-\nabla_{\vw} L = [4, 8]\transpose$ points toward better parameters.
\end{example}

\subsection{The Chain Rule}

\begin{theorem}[Chain Rule for Functions]
\label{thm:chain_rule}
For composite function $h(\vx) = f(g(\vx))$ where $g: \R^n \to \R^m$ and $f: \R^m \to \R$:
\begin{equation}
\frac{\partial h}{\partial x_i} = \sum_{j=1}^m \frac{\partial f}{\partial g_j} \frac{\partial g_j}{\partial x_i}
\end{equation}

In vector form:
\begin{equation}
\nabla_{\vx} h = \mJ_g\transpose \nabla_{\vz} f
\end{equation}
where $\vz = g(\vx)$ and $\mJ_g \in \R^{m \times n}$ is the Jacobian of $g$.
\end{theorem}

\begin{example}[Chain Rule Application]
\label{ex:chain_rule}
For neural network layer: $\vy = \sigma(\mW\vx + \vb)$ where $\sigma$ is applied element-wise.

Let $\vz = \mW\vx + \vb$ (pre-activation). Then:
\begin{equation}
\frac{\partial L}{\partial \vx} = \mW\transpose \left( \frac{\partial L}{\partial \vy} \odot \sigma'(\vz) \right)
\end{equation}

where $\odot$ denotes element-wise multiplication.

\textbf{Concrete example:} For ReLU activation $\sigma(z) = \max(0, z)$:
\begin{equation}
\sigma'(z) = \begin{cases} 1 & \text{if } z > 0 \\ 0 & \text{if } z \leq 0 \end{cases}
\end{equation}

If $\vz = [2.0, -1.0, 0.5]\transpose$, then $\sigma'(\vz) = [1, 0, 1]\transpose$.
\end{example}

\subsection{Jacobian and Hessian Matrices}

\begin{definition}[Jacobian Matrix]
\label{def:jacobian}
For function $\mathbf{f}: \R^n \to \R^m$, the \textbf{Jacobian matrix} is:
\begin{equation}
\mJ_{\mathbf{f}}(\vx) = \begin{bmatrix}
\frac{\partial f_1}{\partial x_1} & \frac{\partial f_1}{\partial x_2} & \cdots & \frac{\partial f_1}{\partial x_n} \\
\frac{\partial f_2}{\partial x_1} & \frac{\partial f_2}{\partial x_2} & \cdots & \frac{\partial f_2}{\partial x_n} \\
\vdots & \vdots & \ddots & \vdots \\
\frac{\partial f_m}{\partial x_1} & \frac{\partial f_m}{\partial x_2} & \cdots & \frac{\partial f_m}{\partial x_n}
\end{bmatrix} \in \R^{m \times n}
\end{equation}
\end{definition}

\begin{definition}[Hessian Matrix]
\label{def:hessian}
For function $f: \R^n \to \R$, the \textbf{Hessian matrix} contains second derivatives:
\begin{equation}
\mH_f(\vx) = \begin{bmatrix}
\frac{\partial^2 f}{\partial x_1^2} & \frac{\partial^2 f}{\partial x_1 \partial x_2} & \cdots \\
\frac{\partial^2 f}{\partial x_2 \partial x_1} & \frac{\partial^2 f}{\partial x_2^2} & \cdots \\
\vdots & \vdots & \ddots
\end{bmatrix} \in \R^{n \times n}
\end{equation}
\end{definition}

The Hessian describes the local curvature of the function. For smooth functions, $\mH$ is symmetric.

\section{Gradient Descent}
\label{sec:gradient_descent}

\subsection{The Gradient Descent Algorithm}

Gradient descent iteratively moves parameters in the direction opposite to the gradient:

\begin{algorithm}[H]
\caption{Gradient Descent}
\label{alg:gradient_descent}
\KwIn{Objective function $f(\vw)$, initial parameters $\vw^{(0)}$, learning rate $\eta$, iterations $T$}
\KwOut{Optimized parameters $\vw^{(T)}$}
\For{$t = 0$ \KwTo $T-1$}{
    Compute gradient: $\mathbf{g}^{(t)} = \nabla f(\vw^{(t)})$ \\
    Update parameters: $\vw^{(t+1)} = \vw^{(t)} - \eta \mathbf{g}^{(t)}$
}
\Return{$\vw^{(T)}$}
\end{algorithm}

\begin{keypoint}
The learning rate $\eta$ controls the step size. Too large: divergence. Too small: slow convergence.
\end{keypoint}

\begin{example}[Gradient Descent on Quadratic]
\label{ex:gd_quadratic}
Minimize $f(w) = w^2$ starting from $w^{(0)} = 3$ with $\eta = 0.1$:
\begin{align}
t=0:& \quad w^{(0)} = 3, \quad g^{(0)} = 2w^{(0)} = 6, \quad w^{(1)} = 3 - 0.1(6) = 2.4 \\
t=1:& \quad w^{(1)} = 2.4, \quad g^{(1)} = 4.8, \quad w^{(2)} = 2.4 - 0.1(4.8) = 1.92 \\
t=2:& \quad w^{(2)} = 1.92, \quad g^{(2)} = 3.84, \quad w^{(3)} = 1.92 - 0.1(3.84) = 1.536
\end{align}

The parameters converge to $w^* = 0$ (the minimum).
\end{example}

\subsection{Stochastic Gradient Descent (SGD)}

For large datasets, computing the full gradient is expensive. SGD approximates the gradient using mini-batches.

\begin{algorithm}[H]
\caption{Stochastic Gradient Descent (SGD)}
\label{alg:sgd}
\KwIn{Dataset $\mathcal{D} = \{(\vx^{(i)}, y^{(i)})\}_{i=1}^N$, batch size $B$, learning rate $\eta$, epochs $E$}
\KwOut{Optimized parameters $\vw$}
Initialize $\vw$ randomly \\
\For{epoch $e = 1$ \KwTo $E$}{
    Shuffle dataset $\mathcal{D}$ \\
    \For{each mini-batch $\mathcal{B} \subset \mathcal{D}$ of size $B$}{
        Compute mini-batch gradient: $\mathbf{g} = \frac{1}{B} \sum_{(\vx, y) \in \mathcal{B}} \nabla_{\vw} L(\vw; \vx, y)$ \\
        Update: $\vw \leftarrow \vw - \eta \mathbf{g}$
    }
}
\Return{$\vw$}
\end{algorithm}

\begin{implementation}
PyTorch SGD implementation:
\begin{lstlisting}[language=Python]
import torch
import torch.nn as nn

# Model and loss
model = nn.Linear(10, 1)
criterion = nn.MSELoss()

# SGD optimizer
optimizer = torch.optim.SGD(model.parameters(), lr=0.01)

# Training loop
for epoch in range(100):
    for x_batch, y_batch in dataloader:
        # Forward pass
        y_pred = model(x_batch)
        loss = criterion(y_pred, y_batch)

        # Backward pass
        optimizer.zero_grad()  # Clear previous gradients
        loss.backward()         # Compute gradients
        optimizer.step()        # Update parameters
\end{lstlisting}
\end{implementation}

\subsection{Momentum}

Momentum accelerates SGD by accumulating a velocity vector:

\begin{algorithm}[H]
\caption{SGD with Momentum}
\label{alg:momentum}
\KwIn{Learning rate $\eta$, momentum coefficient $\beta$ (typically 0.9)}
Initialize velocity $\mathbf{v} = \mathbf{0}$ \\
\For{each iteration}{
    Compute gradient $\mathbf{g} = \nabla_{\vw} L(\vw)$ \\
    Update velocity: $\mathbf{v} \leftarrow \beta \mathbf{v} + \mathbf{g}$ \\
    Update parameters: $\vw \leftarrow \vw - \eta \mathbf{v}$
}
\end{algorithm}

Momentum helps navigate ravines and accelerates convergence in relevant directions.

\subsection{Adam Optimizer}

Adam (Adaptive Moment Estimation) combines momentum with adaptive learning rates:

\begin{algorithm}[H]
\caption{Adam Optimizer}
\label{alg:adam}
\KwIn{Learning rate $\alpha$ (default 0.001), $\beta_1$ = 0.9, $\beta_2$ = 0.999, $\epsilon$ = $10^{-8}$}
Initialize $\mathbf{m}_0 = \mathbf{0}$ (first moment), $\mathbf{v}_0 = \mathbf{0}$ (second moment), $t = 0$ \\
\While{not converged}{
    $t \leftarrow t + 1$ \\
    Compute gradient: $\mathbf{g}_t = \nabla_{\vw} L(\vw_{t-1})$ \\
    Update biased first moment: $\mathbf{m}_t = \beta_1 \mathbf{m}_{t-1} + (1 - \beta_1) \mathbf{g}_t$ \\
    Update biased second moment: $\mathbf{v}_t = \beta_2 \mathbf{v}_{t-1} + (1 - \beta_2) \mathbf{g}_t^2$ \\
    Bias-corrected first moment: $\hat{\mathbf{m}}_t = \mathbf{m}_t / (1 - \beta_1^t)$ \\
    Bias-corrected second moment: $\hat{\mathbf{v}}_t = \mathbf{v}_t / (1 - \beta_2^t)$ \\
    Update parameters: $\vw_t = \vw_{t-1} - \alpha \hat{\mathbf{m}}_t / (\sqrt{\hat{\mathbf{v}}_t} + \epsilon)$
}
\end{algorithm}

Adam is the most commonly used optimizer for training transformers and large language models.

\section{Gradient Computation Complexity}
\label{sec:gradient_complexity}

Understanding the computational and memory costs of gradient computation is essential for training large models efficiently.

\subsection{FLOPs for Gradient Computation}

\begin{keypoint}
Computing gradients via backpropagation requires approximately 2× the FLOPs of the forward pass: 1× for the backward pass itself, plus the original 1× forward pass.
\end{keypoint}

\begin{example}[BERT-base Gradient Computation]
\label{ex:bert_gradient_flops}
For BERT-base (110M parameters, 12 layers, $d_{\text{model}} = 768$) processing sequence length $n = 512$:

\textbf{Forward pass:}
\begin{itemize}
    \item Self-attention: $12 \times 4n^2d = 12 \times 4(512)^2(768) \approx 48$ GFLOPs
    \item Feed-forward: $12 \times 2nd(4d) = 12 \times 2(512)(768)(3072) \approx 36$ GFLOPs
    \item Other operations: $\approx 12$ GFLOPs
    \item \textbf{Total forward: $\approx 96$ GFLOPs}
\end{itemize}

\textbf{Backward pass:}
\begin{itemize}
    \item Gradient computation through each layer: $\approx 96$ GFLOPs
    \item Gradient accumulation for weight updates: $\approx 97$ GFLOPs
    \item \textbf{Total backward: $\approx 193$ GFLOPs}
\end{itemize}

\textbf{Total per training step: $\approx 289$ GFLOPs}

For batch size $B = 32$: $289 \times 32 \approx 9.2$ TFLOPs per batch.
\end{example}

\subsection{Memory Requirements for Activations}

During backpropagation, intermediate activations must be stored for gradient computation.

\begin{definition}[Activation Memory]
\label{def:activation_memory}
For a network with $L$ layers processing batch size $B$, activation memory is:
\begin{equation}
M_{\text{act}} = B \sum_{\ell=1}^L d_\ell
\end{equation}
where $d_\ell$ is the dimension of layer $\ell$'s output.
\end{definition}

\begin{example}[BERT-base Activation Memory]
\label{ex:bert_activation_memory}
For BERT-base with batch size $B = 32$, sequence length $n = 512$, $d = 768$:

\textbf{Per transformer layer:}
\begin{itemize}
    \item Query, Key, Value projections: $3 \times Bnd = 3 \times 32 \times 512 \times 768 \times 4 \text{ bytes} \approx 113$ MB
    \item Attention scores: $B \times h \times n \times n = 32 \times 12 \times 512 \times 512 \times 4 \text{ bytes} \approx 402$ MB
    \item Attention output: $Bnd \approx 38$ MB
    \item Feed-forward intermediate: $B \times n \times 4d \approx 151$ MB
    \item \textbf{Per layer total: $\approx 704$ MB}
\end{itemize}

\textbf{For 12 layers: $704 \times 12 \approx 8.4$ GB}

This excludes gradients and optimizer states!
\end{example}

\subsection{Automatic Differentiation: Forward vs Reverse Mode}

\begin{definition}[Forward Mode AD]
\label{def:forward_mode}
Forward mode computes derivatives by propagating tangent vectors forward through the computational graph. For function $f: \R^n \to \R^m$, computing $\nabla f$ requires $n$ forward passes.
\end{definition}

\begin{definition}[Reverse Mode AD]
\label{def:reverse_mode}
Reverse mode (backpropagation) computes derivatives by propagating adjoints backward. Computing $\nabla f$ requires 1 forward pass + 1 backward pass, regardless of $n$.
\end{definition}

\begin{keypoint}
For neural networks where $n \gg m$ (millions of parameters, one loss), reverse mode is vastly more efficient: $O(1)$ passes vs $O(n)$ passes.
\end{keypoint}

\begin{example}[Forward vs Reverse Mode Comparison]
\label{ex:forward_vs_reverse}
For a network with $n = 10^8$ parameters (100M) and scalar loss ($m = 1$):

\textbf{Forward mode:}
\begin{itemize}
    \item Requires $10^8$ forward passes
    \item Each pass: $\approx 100$ GFLOPs
    \item Total: $10^{10}$ GFLOPs $\approx 10$ PFLOPs
    \item Time on A100 GPU (312 TFLOPS): $\approx 32,000$ seconds $\approx 9$ hours
\end{itemize}

\textbf{Reverse mode (backpropagation):}
\begin{itemize}
    \item Requires 1 forward + 1 backward pass
    \item Total: $\approx 300$ GFLOPs
    \item Time on A100 GPU: $\approx 0.001$ seconds
    \item \textbf{Speedup: $\approx 32$ million×}
\end{itemize}
\end{example}

\subsection{Gradient Checkpointing}

Gradient checkpointing trades computation for memory by recomputing activations during the backward pass.

\begin{algorithm}[H]
\caption{Gradient Checkpointing}
\label{alg:gradient_checkpointing}
\KwIn{Network with $L$ layers, checkpoint every $k$ layers}
\tcp{Forward Pass}
\For{$\ell = 1$ \KwTo $L$}{
    Compute $\vh^{(\ell)} = f^{(\ell)}(\vh^{(\ell-1)})$ \\
    \If{$\ell \bmod k = 0$}{
        Save $\vh^{(\ell)}$ to memory (checkpoint)
    }
}

\tcp{Backward Pass}
\For{$\ell = L$ \KwTo $1$}{
    \If{$\vh^{(\ell)}$ not in memory}{
        Recompute forward from last checkpoint to layer $\ell$
    }
    Compute gradient $\nabla_{\vh^{(\ell-1)}} L$ using $\vh^{(\ell)}$
}
\end{algorithm}

\begin{example}[Checkpointing Trade-off]
\label{ex:checkpointing_tradeoff}
For BERT-base (12 layers) with checkpointing every 3 layers:

\textbf{Without checkpointing:}
\begin{itemize}
    \item Memory: $8.4$ GB (all activations)
    \item Computation: $289$ GFLOPs (1 forward + 1 backward)
\end{itemize}

\textbf{With checkpointing (every 3 layers):}
\begin{itemize}
    \item Memory: $8.4 / 3 \approx 2.8$ GB (only checkpoints)
    \item Computation: $96 + 193 + 72 = 361$ GFLOPs (1 forward + 1 backward + 0.75 forward recompute)
    \item \textbf{Memory reduction: 3×, Computation increase: 1.25×}
\end{itemize}

For GPT-3 (175B parameters), checkpointing is essential to fit in GPU memory.
\end{example}

\section{Backpropagation}
\label{sec:backpropagation}

Backpropagation efficiently computes gradients in neural networks using the chain rule.

\subsection{Computational Graphs}

A computational graph represents the sequence of operations in a neural network. Each node is an operation, and edges carry values/gradients.

\begin{example}[Simple Computational Graph]
\label{ex:comp_graph}
For $L = (y - \hat{y})^2$ where $\hat{y} = w_2 \sigma(w_1 x + b_1) + b_2$:

\textbf{Forward pass:}
\begin{align}
z_1 &= w_1 x + b_1 = 2.0(1.0) + 0.5 = 2.5 \\
a_1 &= \sigma(z_1) = \sigma(2.5) = 0.924 \quad \text{(sigmoid)} \\
z_2 &= w_2 a_1 + b_2 = 1.5(0.924) + 0.3 = 1.686 \\
L &= (y - z_2)^2 = (3.0 - 1.686)^2 = 1.726
\end{align}

\textbf{Backward pass:}
\begin{align}
\frac{\partial L}{\partial z_2} &= 2(z_2 - y) = 2(1.686 - 3.0) = -2.628 \\
\frac{\partial L}{\partial w_2} &= \frac{\partial L}{\partial z_2} \cdot a_1 = -2.628(0.924) = -2.428 \\
\frac{\partial L}{\partial a_1} &= \frac{\partial L}{\partial z_2} \cdot w_2 = -2.628(1.5) = -3.942 \\
\frac{\partial L}{\partial z_1} &= \frac{\partial L}{\partial a_1} \cdot \sigma'(z_1) = -3.942(0.070) = -0.276 \\
\frac{\partial L}{\partial w_1} &= \frac{\partial L}{\partial z_1} \cdot x = -0.276(1.0) = -0.276
\end{align}
\end{example}

\subsection{Backpropagation Algorithm}

\begin{algorithm}[H]
\caption{Backpropagation}
\label{alg:backprop}
\KwIn{Training example $(\vx, y)$, network with $L$ layers}
\KwOut{Gradients $\{\nabla_{\mW^{(\ell)}} L, \nabla_{\vb^{(\ell)}} L\}_{\ell=1}^L$}

\tcp{Forward Pass}
$\vh^{(0)} = \vx$ \\
\For{$\ell = 1$ \KwTo $L$}{
    $\vz^{(\ell)} = \mW^{(\ell)} \vh^{(\ell-1)} + \vb^{(\ell)}$ \\
    $\vh^{(\ell)} = \sigma^{(\ell)}(\vz^{(\ell)})$
}
$\hat{y} = \vh^{(L)}$ \\
Compute loss: $L = \text{Loss}(y, \hat{y})$

\tcp{Backward Pass}
$\boldsymbol{\delta}^{(L)} = \nabla_{\vh^{(L)}} L \odot \sigma'^{(L)}(\vz^{(L)})$ \\
\For{$\ell = L$ \KwTo $1$}{
    $\nabla_{\mW^{(\ell)}} L = \boldsymbol{\delta}^{(\ell)} (\vh^{(\ell-1)})\transpose$ \\
    $\nabla_{\vb^{(\ell)}} L = \boldsymbol{\delta}^{(\ell)}$ \\
    \If{$\ell > 1$}{
        $\boldsymbol{\delta}^{(\ell-1)} = (\mW^{(\ell)})\transpose \boldsymbol{\delta}^{(\ell)} \odot \sigma'^{(\ell-1)}(\vz^{(\ell-1)})$
    }
}
\Return{All gradients}
\end{algorithm}

\begin{keypoint}
Backpropagation computes gradients in $O(n)$ time where $n$ is the number of parameters, compared to $O(n^2)$ for naive methods. This efficiency enables training of billion-parameter models.
\end{keypoint}

\subsection{Why Backpropagation is $O(n)$ Not $O(n^2)$}

\begin{theorem}[Backpropagation Complexity]
\label{thm:backprop_complexity}
For a neural network with $n$ parameters and $m$ operations, backpropagation computes all gradients in $O(m)$ time, where typically $m = O(n)$.
\end{theorem}

\begin{proof}[Intuition]
Each operation in the forward pass corresponds to one gradient computation in the backward pass. The chain rule allows us to reuse intermediate gradients:
\begin{equation}
\frac{\partial L}{\partial w_i} = \frac{\partial L}{\partial z_j} \cdot \frac{\partial z_j}{\partial w_i}
\end{equation}

We compute $\frac{\partial L}{\partial z_j}$ once and reuse it for all parameters that affect $z_j$. This sharing prevents the $O(n^2)$ cost of computing each gradient independently.
\end{proof}

\begin{example}[Complexity Comparison]
\label{ex:complexity_comparison}
For a network with $n = 10^8$ parameters:

\textbf{Naive finite differences:}
\begin{equation}
\frac{\partial L}{\partial w_i} \approx \frac{L(w_i + \epsilon) - L(w_i)}{\epsilon}
\end{equation}
Requires $n$ forward passes: $O(n \cdot m) = O(n^2)$ operations.

\textbf{Backpropagation:}
\begin{itemize}
    \item Forward pass: $O(m) = O(n)$ operations
    \item Backward pass: $O(m) = O(n)$ operations
    \item Total: $O(n)$ operations
\end{itemize}

\textbf{Speedup: $O(n) = 10^8$×}
\end{example}

\section{Optimizer Memory Requirements}
\label{sec:optimizer_memory}

Different optimizers have vastly different memory requirements, which becomes critical for large models.

\subsection{Memory Comparison by Optimizer}

\begin{table}[h]
\centering
\begin{tabular}{lcc}
\toprule
\textbf{Optimizer} & \textbf{Memory per Parameter} & \textbf{Total Memory Factor} \\
\midrule
SGD (no momentum) & 4 bytes (fp32) & 1× \\
SGD with momentum & 8 bytes (param + velocity) & 2× \\
Adam & 16 bytes (param + 2 moments) & 4× \\
Adam (mixed precision) & 10 bytes (fp16 param + fp32 master + 2 moments) & 2.5× \\
\bottomrule
\end{tabular}
\caption{Memory requirements per parameter for different optimizers}
\label{tab:optimizer_memory}
\end{table}

\begin{example}[BERT-base Memory Requirements]
\label{ex:bert_memory}
For BERT-base with 110M parameters:

\textbf{Model parameters:}
\begin{itemize}
    \item FP32: $110 \times 10^6 \times 4 \text{ bytes} = 440$ MB
    \item FP16: $110 \times 10^6 \times 2 \text{ bytes} = 220$ MB
\end{itemize}

\textbf{SGD with momentum:}
\begin{itemize}
    \item Parameters: 440 MB
    \item Momentum buffer: 440 MB
    \item \textbf{Total: 880 MB}
\end{itemize}

\textbf{Adam optimizer:}
\begin{itemize}
    \item Parameters: 440 MB
    \item First moment ($\mathbf{m}$): 440 MB
    \item Second moment ($\mathbf{v}$): 440 MB
    \item Gradients: 440 MB
    \item \textbf{Total: 1,760 MB $\approx 1.7$ GB}
\end{itemize}

\textbf{Adam with mixed precision:}
\begin{itemize}
    \item FP16 parameters: 220 MB
    \item FP32 master copy: 440 MB
    \item FP32 first moment: 440 MB
    \item FP32 second moment: 440 MB
    \item FP16 gradients: 220 MB
    \item \textbf{Total: 1,760 MB $\approx 1.7$ GB}
\end{itemize}

Note: Mixed precision doesn't reduce optimizer memory, but enables larger batch sizes.
\end{example}

\begin{example}[GPT-3 Memory Requirements]
\label{ex:gpt3_memory}
For GPT-3 (175B parameters) with Adam optimizer:

\textbf{Model + optimizer states:}
\begin{itemize}
    \item Parameters (FP16): $175 \times 10^9 \times 2 = 350$ GB
    \item Master copy (FP32): $175 \times 10^9 \times 4 = 700$ GB
    \item First moment (FP32): 700 GB
    \item Second moment (FP32): 700 GB
    \item Gradients (FP16): 350 GB
    \item \textbf{Total: 2,800 GB $\approx 2.8$ TB}
\end{itemize}

This requires distributed training across multiple GPUs. With 8× A100 GPUs (80 GB each = 640 GB total), we need model parallelism and optimizer state sharding (e.g., ZeRO optimizer).
\end{example}

\subsection{Impact on GPU Memory Budget}

\begin{keypoint}
For large models, optimizer states often consume more memory than the model itself. Adam uses 4× parameter memory, leaving less room for batch size and activations.
\end{keypoint}

\begin{example}[Memory Budget Breakdown]
\label{ex:memory_budget}
Training BERT-base on A100 GPU (80 GB memory):

\textbf{Memory allocation:}
\begin{itemize}
    \item Model parameters: 0.44 GB
    \item Optimizer states (Adam): 1.32 GB
    \item Activations (batch size 32): 8.4 GB
    \item Gradients: 0.44 GB
    \item Framework overhead: $\approx 2$ GB
    \item \textbf{Total: $\approx 12.6$ GB}
\end{itemize}

\textbf{Remaining: 67.4 GB} available for larger batch sizes or longer sequences.

With batch size 256: Activations $\approx 67$ GB, total $\approx 71$ GB (fits comfortably).
\end{example}

\section{Learning Rate Schedules}
\label{sec:lr_schedules}

Learning rate schedules adjust $\eta$ during training to improve convergence.

\subsection{Learning Rate Impact on Convergence and GPU Utilization}

\begin{keypoint}
Learning rate affects both convergence speed and hardware efficiency. Larger learning rates enable larger batch sizes, improving GPU utilization.
\end{keypoint}

\begin{example}[Learning Rate vs Convergence Speed]
\label{ex:lr_convergence}
Training BERT-base on 1M examples:

\begin{table}[h]
\centering
\begin{tabular}{cccc}
\toprule
\textbf{Learning Rate} & \textbf{Steps to Converge} & \textbf{Wall Time} & \textbf{Final Loss} \\
\midrule
$1 \times 10^{-5}$ & 100,000 & 12 hours & 1.85 \\
$1 \times 10^{-4}$ & 30,000 & 3.6 hours & 1.82 \\
$5 \times 10^{-4}$ & 15,000 & 1.8 hours & 1.81 \\
$1 \times 10^{-3}$ & 20,000 & 2.4 hours & 1.83 \\
$5 \times 10^{-3}$ & Diverges & - & - \\
\bottomrule
\end{tabular}
\caption{Learning rate impact on BERT-base training}
\end{table}

Optimal learning rate ($5 \times 10^{-4}$) achieves 6.7× faster convergence than conservative rate.
\end{example}

\subsection{Learning Rate Scaling with Batch Size}

\begin{theorem}[Linear Scaling Rule]
\label{thm:linear_scaling}
When increasing batch size by factor $k$, scale learning rate by $k$ to maintain convergence behavior:
\begin{equation}
\eta_{\text{new}} = k \cdot \eta_{\text{base}}
\end{equation}

This holds approximately for $k \leq 8$. For larger $k$, use gradual warmup.
\end{theorem}

\begin{example}[Batch Size and Learning Rate Scaling]
\label{ex:batch_lr_scaling}
Training BERT-base with different batch sizes:

\begin{table}[h]
\centering
\begin{tabular}{ccccc}
\toprule
\textbf{Batch Size} & \textbf{Learning Rate} & \textbf{GPU Util.} & \textbf{Steps/sec} & \textbf{Samples/sec} \\
\midrule
32 & $5 \times 10^{-4}$ & 45\% & 2.1 & 67 \\
64 & $1 \times 10^{-3}$ & 68\% & 1.8 & 115 \\
128 & $2 \times 10^{-3}$ & 85\% & 1.4 & 179 \\
256 & $4 \times 10^{-3}$ & 92\% & 1.0 & 256 \\
512 & $8 \times 10^{-3}$ & 95\% & 0.6 & 307 \\
\bottomrule
\end{tabular}
\caption{Batch size impact on GPU utilization (A100 GPU)}
\end{table}

Larger batches improve GPU utilization but require proportionally larger learning rates. Throughput increases 4.6× from batch 32 to 512.
\end{example}

\subsection{Practical Learning Rates for Transformers}

\begin{table}[h]
\centering
\begin{tabular}{lcc}
\toprule
\textbf{Model} & \textbf{Batch Size} & \textbf{Peak Learning Rate} \\
\midrule
BERT-base & 256 & $1 \times 10^{-4}$ \\
BERT-large & 256 & $5 \times 10^{-5}$ \\
GPT-2 (117M) & 512 & $2.5 \times 10^{-4}$ \\
GPT-2 (1.5B) & 512 & $1.5 \times 10^{-4}$ \\
GPT-3 (175B) & 3.2M & $6 \times 10^{-5}$ \\
T5-base & 128 & $1 \times 10^{-4}$ \\
T5-11B & 2048 & $1 \times 10^{-4}$ \\
\bottomrule
\end{tabular}
\caption{Typical learning rates for transformer models}
\label{tab:transformer_lrs}
\end{table}

\begin{keypoint}
Larger models generally require smaller learning rates for stability. GPT-3 uses $6 \times 10^{-5}$ despite massive batch size of 3.2M tokens.
\end{keypoint}

\subsection{Common Schedules}

\textbf{Step Decay:}
\begin{equation}
\eta_t = \eta_0 \gamma^{\lfloor t/s \rfloor}
\end{equation}
where $\gamma < 1$ (e.g., 0.1) and $s$ is step size (e.g., every 10 epochs).

\textbf{Exponential Decay:}
\begin{equation}
\eta_t = \eta_0 e^{-\lambda t}
\end{equation}

\textbf{Cosine Annealing:}
\begin{equation}
\eta_t = \eta_{\min} + \frac{1}{2}(\eta_{\max} - \eta_{\min})\left(1 + \cos\left(\frac{t\pi}{T}\right)\right)
\end{equation}

\textbf{Warmup + Decay (Transformers):}
\begin{equation}
\eta_t = \frac{d_{\text{model}}^{-0.5}}{\max(t, \text{warmup\_steps}^{-0.5})} \cdot \min(t^{-0.5}, t \cdot \text{warmup\_steps}^{-1.5})
\end{equation}

The warmup phase prevents instability in early training of transformers.

\section{Hardware Considerations for Gradient Computation}
\label{sec:hardware_gradients}

Modern deep learning relies on specialized hardware for efficient gradient computation.

\subsection{Gradient Computation on GPUs}

GPUs excel at gradient computation due to massive parallelism in matrix operations.

\begin{example}[GPU vs CPU Gradient Computation]
\label{ex:gpu_cpu_gradients}
Computing gradients for BERT-base (110M parameters) on one training batch:

\textbf{NVIDIA A100 GPU:}
\begin{itemize}
    \item Forward pass: 0.31 ms (96 GFLOPs ÷ 312 TFLOPS)
    \item Backward pass: 0.62 ms (193 GFLOPs ÷ 312 TFLOPS)
    \item \textbf{Total: 0.93 ms per batch}
    \item Throughput: 1,075 batches/second
\end{itemize}

\textbf{Intel Xeon CPU (32 cores):}
\begin{itemize}
    \item Forward pass: 45 ms (96 GFLOPs ÷ 2.1 TFLOPS)
    \item Backward pass: 90 ms (193 GFLOPs ÷ 2.1 TFLOPS)
    \item \textbf{Total: 135 ms per batch}
    \item Throughput: 7.4 batches/second
\end{itemize}

\textbf{GPU speedup: 145×}
\end{example}

\subsection{Mixed Precision Training}

Mixed precision uses FP16 for computation and FP32 for accumulation, reducing memory and increasing speed.

\begin{algorithm}[H]
\caption{Mixed Precision Training}
\label{alg:mixed_precision}
\KwIn{Model parameters $\vw$ (FP32 master copy)}
\For{each training step}{
    Convert $\vw$ to FP16: $\vw_{16} = \text{FP16}(\vw)$ \\
    Forward pass in FP16: $\hat{y} = f(\vx; \vw_{16})$ \\
    Compute loss: $L = \text{Loss}(y, \hat{y})$ \\
    Scale loss: $L_{\text{scaled}} = s \cdot L$ (prevent underflow) \\
    Backward pass in FP16: $\mathbf{g}_{16} = \nabla_{\vw_{16}} L_{\text{scaled}}$ \\
    Unscale gradients: $\mathbf{g}_{16} = \mathbf{g}_{16} / s$ \\
    Convert to FP32: $\mathbf{g} = \text{FP32}(\mathbf{g}_{16})$ \\
    Update FP32 master: $\vw \leftarrow \vw - \eta \mathbf{g}$
}
\end{algorithm}

\begin{example}[Mixed Precision Impact]
\label{ex:mixed_precision_impact}
Training BERT-base on A100 GPU:

\textbf{FP32 training:}
\begin{itemize}
    \item Forward + backward: 0.93 ms
    \item Memory: 12.6 GB
    \item Max batch size: 32
    \item Throughput: 34,400 samples/sec
\end{itemize}

\textbf{Mixed precision (FP16) training:}
\begin{itemize}
    \item Forward + backward: 0.48 ms (1.94× faster)
    \item Memory: 8.2 GB (35\% reduction)
    \item Max batch size: 64
    \item Throughput: 133,300 samples/sec (3.87× faster)
\end{itemize}

Mixed precision provides 1.94× computational speedup and enables 2× larger batches, yielding 3.87× total throughput improvement.
\end{example}

\subsection{Gradient Accumulation}

Gradient accumulation simulates large batch sizes by accumulating gradients over multiple forward-backward passes.

\begin{algorithm}[H]
\caption{Gradient Accumulation}
\label{alg:gradient_accumulation}
\KwIn{Desired batch size $B$, physical batch size $b$, accumulation steps $k = B/b$}
Initialize gradients: $\mathbf{g}_{\text{acc}} = \mathbf{0}$ \\
\For{$i = 1$ \KwTo $k$}{
    Sample mini-batch $\mathcal{B}_i$ of size $b$ \\
    Forward pass: $L_i = \text{Loss}(\mathcal{B}_i)$ \\
    Backward pass: $\mathbf{g}_i = \nabla L_i$ \\
    Accumulate: $\mathbf{g}_{\text{acc}} \leftarrow \mathbf{g}_{\text{acc}} + \mathbf{g}_i$
}
Average: $\mathbf{g}_{\text{acc}} \leftarrow \mathbf{g}_{\text{acc}} / k$ \\
Update parameters: $\vw \leftarrow \vw - \eta \mathbf{g}_{\text{acc}}$ \\
Clear gradients: $\mathbf{g}_{\text{acc}} = \mathbf{0}$
\end{algorithm}

\begin{example}[Gradient Accumulation for Large Batches]
\label{ex:gradient_accumulation}
Training GPT-2 (1.5B parameters) on single A100 GPU (80 GB):

\textbf{Without accumulation:}
\begin{itemize}
    \item Max batch size: 4 (memory limit)
    \item Update frequency: every 4 samples
    \item Training unstable (batch too small)
\end{itemize}

\textbf{With gradient accumulation (32 steps):}
\begin{itemize}
    \item Physical batch size: 4
    \item Effective batch size: $4 \times 32 = 128$
    \item Update frequency: every 128 samples
    \item Memory: same as batch size 4
    \item Training stable and efficient
\end{itemize}

Trade-off: 32× more forward-backward passes per update, but enables training large models on limited hardware.
\end{example}

\subsection{Distributed Gradient Synchronization}

For multi-GPU training, gradients must be synchronized across devices.

\begin{algorithm}[H]
\caption{Data Parallel Training with Gradient Synchronization}
\label{alg:data_parallel}
\KwIn{$N$ GPUs, global batch size $B$, local batch size $b = B/N$}
\For{each GPU $i = 1, \ldots, N$ in parallel}{
    Sample local mini-batch $\mathcal{B}_i$ of size $b$ \\
    Forward pass: $L_i = \text{Loss}(\mathcal{B}_i)$ \\
    Backward pass: $\mathbf{g}_i = \nabla L_i$
}
All-reduce gradients: $\mathbf{g} = \frac{1}{N} \sum_{i=1}^N \mathbf{g}_i$ \\
\For{each GPU $i = 1, \ldots, N$ in parallel}{
    Update local parameters: $\vw_i \leftarrow \vw_i - \eta \mathbf{g}$
}
\end{algorithm}

\begin{example}[Distributed Training Efficiency]
\label{ex:distributed_efficiency}
Training BERT-base on 8× A100 GPUs with NVLink:

\textbf{Single GPU baseline:}
\begin{itemize}
    \item Batch size: 32
    \item Time per step: 0.93 ms
    \item Throughput: 34,400 samples/sec
\end{itemize}

\textbf{8 GPUs (data parallel):}
\begin{itemize}
    \item Global batch size: 256
    \item Time per step: 0.93 ms (computation) + 0.12 ms (communication)
    \item Total: 1.05 ms
    \item Throughput: 243,800 samples/sec
    \item \textbf{Scaling efficiency: 243,800 / (8 × 34,400) = 88.6\%}
\end{itemize}

Communication overhead is 11.4\% due to gradient all-reduce. NVLink (600 GB/s) enables efficient synchronization.
\end{example}

\begin{keypoint}
For large models, gradient synchronization can become a bottleneck. Techniques like gradient compression, ZeRO optimizer, and pipeline parallelism reduce communication overhead.
\end{keypoint}

\section{Exercises}

\begin{exercise}
Compute the gradient of $f(\vw) = \vw\transpose \mA \vw + \vb\transpose \vw + c$ where $\mA \in \R^{n \times n}$ is symmetric, $\vw, \vb \in \R^n$, and $c \in \R$.
\end{exercise}

\begin{exercise}
Implement backpropagation for a 2-layer network with ReLU activation. Given input $\vx = [1.0, 0.5]\transpose$, weights $\mW^{(1)} \in \R^{3 \times 2}$, $\mW^{(2)} \in \R^{1 \times 3}$, and target $y = 2.0$, compute all gradients.
\end{exercise}

\begin{exercise}
For Adam optimizer with $\beta_1 = 0.9$, $\beta_2 = 0.999$, $\alpha = 0.001$:
\begin{enumerate}
    \item Why is bias correction necessary?
    \item What are the effective learning rates after steps $t = 1, 10, 100, 1000$?
    \item How does Adam handle sparse gradients compared to SGD?
\end{enumerate}
\end{exercise}

\begin{exercise}
A transformer is trained with learning rate warmup over 4000 steps, then inverse square root decay. If $d_{\text{model}} = 512$:
\begin{enumerate}
    \item Plot the learning rate schedule for 100,000 steps
    \item What is the learning rate at step 1, 4000, and 10,000?
    \item Why is warmup beneficial for transformer training?
\end{enumerate}
\end{exercise}


\begin{exercise}
Calculate the memory requirements for training GPT-2 (1.5B parameters) with Adam optimizer:
\begin{enumerate}
    \item Model parameters in FP16
    \item Optimizer states (FP32 master copy + 2 moments)
    \item Gradients in FP16
    \item Total memory for model + optimizer
    \item How many A100 GPUs (80 GB each) are needed?
\end{enumerate}
\end{exercise}

\begin{exercise}
For BERT-base processing sequence length 512 with batch size 64:
\begin{enumerate}
    \item Calculate total FLOPs for one training step (forward + backward)
    \item Estimate time per step on A100 GPU (312 TFLOPS)
    \item How does mixed precision (FP16) affect throughput?
    \item What is the maximum batch size that fits in 80 GB memory?
\end{enumerate}
\end{exercise}

\begin{exercise}
Compare gradient computation methods for a network with $10^7$ parameters:
\begin{enumerate}
    \item How many forward passes does finite differences require?
    \item How many passes does backpropagation require?
    \item If one forward pass takes 10 ms, compare total time
    \item Why is reverse mode AD preferred over forward mode?
\end{enumerate}
\end{exercise}

\begin{exercise}
Implement gradient checkpointing for a 24-layer transformer:
\begin{enumerate}
    \item Without checkpointing, how much activation memory is needed?
    \item With checkpointing every 6 layers, what is the memory reduction?
    \item What is the computational overhead (extra forward passes)?
    \item At what model size does checkpointing become necessary?
\end{enumerate}
\end{exercise}

\begin{exercise}
Analyze distributed training efficiency for 8 GPUs:
\begin{enumerate}
    \item If gradient all-reduce takes 15 ms and computation takes 100 ms, what is the scaling efficiency?
    \item How does batch size affect communication overhead?
    \item Compare ring all-reduce vs tree all-reduce for 64 GPUs
    \item When does gradient compression become beneficial?
\end{enumerate}
\end{exercise}
