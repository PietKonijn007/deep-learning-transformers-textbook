\chapter{Attention Mechanisms: Fundamentals}
\label{chap:attention_fundamentals}

\section*{Chapter Overview}

Attention mechanisms revolutionized sequence modeling by allowing models to focus on relevant parts of the input when producing each output. This chapter introduces attention from first principles, developing the query-key-value paradigm that underpins modern transformers.

Attention solves a fundamental limitation of RNN encoder-decoder models: compressing entire input sequence into single fixed-size vector. Instead, attention computes dynamic, context-dependent representations by weighted combination of all input positions.

\subsection*{Learning Objectives}

\begin{enumerate}
    \item Understand the motivation for attention in sequence-to-sequence models
    \item Master the query-key-value attention paradigm
    \item Implement additive (Bahdanau) and multiplicative (Luong) attention
    \item Understand scaled dot-product attention
    \item Compute attention weights and apply to values
    \item Visualize and interpret attention distributions
\end{enumerate}

\section{Motivation: The Seq2Seq Bottleneck}
\label{sec:seq2seq_bottleneck}

\subsection{RNN Encoder-Decoder Architecture}

\textbf{Problem setup:} Translate input sequence $\vx_1, \ldots, \vx_n$ to output sequence $\vy_1, \ldots, \vy_m$

\textbf{Standard approach (pre-attention):}
\begin{enumerate}
    \item \textbf{Encoder RNN:} Process input, produce final hidden state $\mathbf{c}$ (context vector)
    \begin{equation}
    \vh_t^{\text{enc}} = \text{RNN}(\vx_t, \vh_{t-1}^{\text{enc}}), \quad \mathbf{c} = \vh_n^{\text{enc}}
    \end{equation}

    \item \textbf{Decoder RNN:} Generate output conditioned on $\mathbf{c}$
    \begin{equation}
    \vh_t^{\text{dec}} = \text{RNN}([\vy_{t-1}, \mathbf{c}], \vh_{t-1}^{\text{dec}})
    \end{equation}
\end{enumerate}

\textbf{Bottleneck:} Entire input sequence compressed into single fixed-size vector $\mathbf{c}$! 
\begin{itemize}
    \item Long sequences: information loss
    \item All input words contribute equally
    \item Performance degrades with sequence length
\end{itemize}

\subsection{Attention Solution}

\textbf{Key insight:} When generating output word $\vy_t$, different input words have different relevance.

\textbf{Attention mechanism:}
\begin{itemize}
    \item Compute \textbf{context vector} $\mathbf{c}_t$ for each output position $t$
    \item $\mathbf{c}_t$ is weighted sum of all encoder hidden states
    \item Weights reflect relevance of each input to current output
\end{itemize}

\begin{example}[Translation with Attention]
\label{ex:translation_attention}
Translate "The cat sat on the mat" to "Le chat était assis sur le tapis"

When generating "chat" (cat), attention should focus on "cat" in input.

When generating "assis" (sat), attention should focus on "sat".

Attention weights adapt dynamically based on what decoder is generating!
\end{example}

\section{Additive Attention (Bahdanau)}
\label{sec:additive_attention}

\begin{definition}[Bahdanau Attention]
\label{def:bahdanau_attention}
Given:
\begin{itemize}
    \item Encoder hidden states: $\vh_1, \ldots, \vh_n \in \R^{d_h}$
    \item Decoder hidden state at time $t$: $\mathbf{s}_t \in \R^{d_s}$
\end{itemize}

\textbf{Step 1: Compute alignment scores}
\begin{equation}
e_{t,i} = \mathbf{v}\transpose \tanh(\mW_1 \mathbf{s}_t + \mW_2 \vh_i)
\end{equation}
where $\mW_1 \in \R^{d_a \times d_s}$, $\mW_2 \in \R^{d_a \times d_h}$, $\mathbf{v} \in \R^{d_a}$, and $d_a$ is attention dimension.

\textbf{Step 2: Compute attention weights (softmax)}
\begin{equation}
\alpha_{t,i} = \frac{\exp(e_{t,i})}{\sum_{j=1}^{n} \exp(e_{t,j})}
\end{equation}

\textbf{Step 3: Compute context vector}
\begin{equation}
\mathbf{c}_t = \sum_{i=1}^{n} \alpha_{t,i} \vh_i
\end{equation}

\textbf{Step 4: Use in decoder}
\begin{equation}
\mathbf{s}_t = \text{RNN}([\vy_{t-1}, \mathbf{c}_t], \mathbf{s}_{t-1})
\end{equation}
\end{definition}

\begin{keypoint}
Attention weights $\alpha_{t,i}$ form probability distribution: $\alpha_{t,i} \geq 0$ and $\sum_{i=1}^n \alpha_{t,i} = 1$.
Context vector $\mathbf{c}_t$ is weighted average of encoder states.
\end{keypoint}

\begin{example}[Bahdanau Attention Computation]
\label{ex:bahdanau_computation}
Setup:
\begin{itemize}
    \item Encoder hidden states: $\vh_1, \vh_2, \vh_3 \in \R^{4}$
    \item Decoder state: $\mathbf{s}_2 \in \R^{4}$
    \item Attention dimension: $d_a = 3$
\end{itemize}

\textbf{Step 1:} Compute scores for each encoder position
\begin{align}
e_{2,1} &= \mathbf{v}\transpose \tanh(\mW_1 \mathbf{s}_2 + \mW_2 \vh_1) \in \R \\
e_{2,2} &= \mathbf{v}\transpose \tanh(\mW_1 \mathbf{s}_2 + \mW_2 \vh_2) \in \R \\
e_{2,3} &= \mathbf{v}\transpose \tanh(\mW_1 \mathbf{s}_2 + \mW_2 \vh_3) \in \R
\end{align}

Suppose: $e_{2,1} = 0.8$, $e_{2,2} = 2.1$, $e_{2,3} = 0.5$

\textbf{Step 2:} Softmax to get weights
\begin{align}
\sum_j \exp(e_{2,j}) &= \exp(0.8) + \exp(2.1) + \exp(0.5) \\
&\approx 2.23 + 8.17 + 1.65 = 12.05 \\
\alpha_{2,1} &= \frac{2.23}{12.05} \approx 0.185 \\
\alpha_{2,2} &= \frac{8.17}{12.05} \approx 0.678 \\
\alpha_{2,3} &= \frac{1.65}{12.05} \approx 0.137
\end{align}

Most attention (67.8\%) on position 2!

\textbf{Step 3:} Compute context
\begin{equation}
\mathbf{c}_2 = 0.185 \vh_1 + 0.678 \vh_2 + 0.137 \vh_3 \in \R^{4}
\end{equation}
\end{example}

\section{Scaled Dot-Product Attention}
\label{sec:scaled_dot_product}

\begin{definition}[Scaled Dot-Product Attention]
\label{def:scaled_dot_product}
Given queries $\mQ \in \R^{m \times d_k}$, keys $\mK \in \R^{n \times d_k}$, values $\mV \in \R^{n \times d_v}$:

\textbf{Step 1: Compute attention scores}
\begin{equation}
\mE = \mQ \mK\transpose \in \R^{m \times n}
\end{equation}
Score $e_{i,j}$ measures compatibility of query $i$ with key $j$.

\textbf{Step 2: Scale by $\sqrt{d_k}$}
\begin{equation}
\mE_{\text{scaled}} = \frac{\mQ \mK\transpose}{\sqrt{d_k}}
\end{equation}

\textbf{Step 3: Softmax over keys (row-wise)}
\begin{equation}
\mA = \text{softmax}\left(\frac{\mQ \mK\transpose}{\sqrt{d_k}}\right) \in \R^{m \times n}
\end{equation}

\textbf{Step 4: Apply attention to values}
\begin{equation}
\text{Attention}(\mQ, \mK, \mV) = \mA \mV \in \R^{m \times d_v}
\end{equation}
\end{definition}

\textbf{Complete formula:}
\begin{equation}
\text{Attention}(\mQ, \mK, \mV) = \text{softmax}\left(\frac{\mQ \mK\transpose}{\sqrt{d_k}}\right) \mV
\end{equation}

\begin{keypoint}
\textbf{Why scale by $\sqrt{d_k}$?}

For large $d_k$, dot products grow large in magnitude, pushing softmax into regions with tiny gradients. Scaling keeps values moderate, maintaining good gradient flow.

If $\mQ$ and $\mK$ have unit variance elements:
\begin{equation}
\text{Var}(\vq\transpose \vk) = d_k \quad \Rightarrow \quad \text{Var}\left(\frac{\vq\transpose \vk}{\sqrt{d_k}}\right) = 1
\end{equation}
\end{keypoint}

\begin{example}[Scaled Dot-Product Computation]
\label{ex:scaled_dot_product}
Query-key-value dimensions: $d_k = 4$, $d_v = 5$

Single query attending to 3 keys:
\begin{equation}
\vq = \begin{bmatrix} 1.0 \\ 0.5 \\ -0.3 \\ 0.8 \end{bmatrix}, \quad
\mK = \begin{bmatrix}
0.8 & 0.2 & -0.1 & 0.5 \\
0.3 & 0.7 & 0.4 & -0.2 \\
-0.5 & 0.1 & 0.9 & 0.6
\end{bmatrix}
\end{equation}

\textbf{Step 1:} Compute dot products
\begin{align}
\vq\transpose \vk_1 &= 1.0(0.8) + 0.5(0.2) + (-0.3)(-0.1) + 0.8(0.5) = 1.33 \\
\vq\transpose \vk_2 &= 1.0(0.3) + 0.5(0.7) + (-0.3)(0.4) + 0.8(-0.2) = 0.43 \\
\vq\transpose \vk_3 &= 1.0(-0.5) + 0.5(0.1) + (-0.3)(0.9) + 0.8(0.6) = -0.22
\end{align}

\textbf{Step 2:} Scale by $\sqrt{d_k} = \sqrt{4} = 2$
\begin{equation}
\text{scores} = [0.665, 0.215, -0.110]
\end{equation}

\textbf{Step 3:} Softmax
\begin{equation}
\sum_j \exp(\text{score}_j) \approx 1.95 + 1.24 + 0.90 = 4.09
\end{equation}
\begin{equation}
\boldsymbol{\alpha} = [0.477, 0.303, 0.220]
\end{equation}

\textbf{Step 4:} Apply to values (suppose $\mV \in \R^{3 \times 5}$)
\begin{equation}
\text{output} = 0.477 \vv_1 + 0.303 \vv_2 + 0.220 \vv_3 \in \R^{5}
\end{equation}
\end{example}

\section{Query-Key-Value Paradigm}
\label{sec:qkv_paradigm}

\subsection{Intuition}

\textbf{Analogy:} Information retrieval system
\begin{itemize}
    \item \textbf{Query ($\vq$):} What I'm looking for
    \item \textbf{Keys ($\vk_i$):} Indexed content descriptions
    \item \textbf{Values ($\vv_i$):} Actual content to retrieve
\end{itemize}

\textbf{Process:}
\begin{enumerate}
    \item Compare query to all keys (compute similarity)
    \item Convert similarities to weights (softmax)
    \item Retrieve weighted combination of values
\end{enumerate}

\subsection{Projecting to QKV}

In transformers, $\mQ$, $\mK$, $\mV$ are computed from inputs via learned projections:

\begin{align}
\mQ &= \mX \mW^Q && \mW^Q \in \R^{d_{\text{model}} \times d_k} \\
\mK &= \mX \mW^K && \mW^K \in \R^{d_{\text{model}} \times d_k} \\
\mV &= \mX \mW^V && \mW^V \in \R^{d_{\text{model}} \times d_v}
\end{align}

where $\mX \in \R^{n \times d_{\text{model}}}$ is the input.

\begin{example}[QKV Projection]
\label{ex:qkv_projection}
Input: Sequence of 5 tokens, each $d_{\text{model}} = 512$ dimensions
\begin{equation}
\mX \in \R^{5 \times 512}
\end{equation}

Project to $d_k = d_v = 64$:
\begin{align}
\mQ &= \mX \mW^Q \in \R^{5 \times 64} \quad (\mW^Q \in \R^{512 \times 64}) \\
\mK &= \mX \mW^K \in \R^{5 \times 64} \quad (\mW^K \in \R^{512 \times 64}) \\
\mV &= \mX \mW^V \in \R^{5 \times 64} \quad (\mW^V \in \R^{512 \times 64})
\end{align}

\textbf{Attention computation:}
\begin{equation}
\underbrace{\text{softmax}\left(\frac{\mQ \mK\transpose}{\sqrt{64}}\right)}_{\R^{5 \times 5}} \underbrace{\mV}_{\R^{5 \times 64}} = \underbrace{\text{Output}}_{\R^{5 \times 64}}
\end{equation}

Attention matrix $\mA \in \R^{5 \times 5}$: entry $(i,j)$ is attention from position $i$ to position $j$.
\end{example}

\section{Attention Variants}
\label{sec:attention_variants}

\subsection{Self-Attention vs Cross-Attention}

\textbf{Self-Attention:} $\mQ$, $\mK$, $\mV$ all from same source
\begin{equation}
\mQ = \mK = \mV = \mX \mW
\end{equation}
Used in: Transformer encoder, BERT

\textbf{Cross-Attention:} Queries from one source, keys and values from another
\begin{equation}
\mQ = \mX_{\text{dec}} \mW^Q, \quad \mK = \mV = \mX_{\text{enc}} \mW^{K/V}
\end{equation}
Used in: Transformer decoder (attending to encoder output)

\subsection{Masked Attention}

For autoregressive models (GPT), prevent attending to future positions:
\begin{equation}
\text{Attention}(\mQ, \mK, \mV) = \text{softmax}\left(\frac{\mQ \mK\transpose + \mM}{\sqrt{d_k}}\right) \mV
\end{equation}
where mask $\mM_{ij} = -\infty$ if $j > i$, else $\mM_{ij} = 0$.

After softmax, $\exp(-\infty) = 0$, so no attention to future!

\section{Exercises}

\begin{exercise}
Compute Bahdanau attention for sequence length 4, decoder state dim 3, attention dim 2. Given specific $\mW_1$, $\mW_2$, $\mathbf{v}$, encoder states, and decoder state, calculate all attention weights.
\end{exercise}

\begin{exercise}
For scaled dot-product attention with $\mQ \in \R^{10 \times 64}$, $\mK \in \R^{20 \times 64}$, $\mV \in \R^{20 \times 128}$: (1) What is output dimension? (2) What is attention matrix shape? (3) How many FLOPs for computing $\mQ \mK\transpose$?
\end{exercise}

\begin{exercise}
Show that without scaling, for $d_k = 64$ and unit variance elements, dot products have variance 64. Demonstrate numerically how this affects softmax gradients.
\end{exercise}

\begin{exercise}
Implement scaled dot-product attention in PyTorch. Test with sequences of length 5 and 10, dimensions $d_k = 32$, $d_v = 48$. Visualize attention weights as heatmap.
\end{exercise}

