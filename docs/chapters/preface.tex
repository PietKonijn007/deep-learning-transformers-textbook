\chapter*{Preface}
\addcontentsline{toc}{chapter}{Preface}

This textbook emerged from the recognition that while deep learning has revolutionized artificial intelligence, there exists a gap between introductory treatments and the advanced mathematical and practical knowledge required to design, implement, and deploy modern transformer-based systems. The remarkable success of large language models, vision transformers, and multimodal systems has created urgent demand for comprehensive educational materials that bridge theory and practice.

\section*{Who This Book Is For}

This book is designed for graduate students in computer science, electrical engineering, mathematics, and related fields who seek a rigorous understanding of deep learning and transformer architectures. We assume readers have:

\begin{itemize}
    \item Strong foundations in linear algebra, including matrix operations, eigendecompositions, and vector spaces
    \item Solid understanding of multivariable calculus, including gradients, partial derivatives, and the chain rule
    \item Basic probability theory and statistics
    \item Programming experience, preferably in Python
    \item Familiarity with machine learning concepts at an introductory level
\end{itemize}

The book is equally valuable for academic researchers, industry practitioners, PhD students, and engineers building production systems based on transformer models.

\section*{Philosophy and Approach}

Our pedagogical philosophy emphasizes four key principles:

\subsection*{1. Mathematical Rigor with Geometric Intuition}
We provide complete mathematical derivations but always accompany them with geometric interpretations and intuitive explanations. Every matrix operation includes explicit dimension annotations.

\subsection*{2. Theory Grounded in Practice}
Abstract concepts are immediately connected to practical implementations. We show how mathematical operations map to efficient tensor computations on GPUs and discuss memory requirements, computational complexity, and optimization strategies.

\subsection*{3. Progressive Complexity}
The book builds knowledge systematically, with early chapters establishing mathematical foundations that are repeatedly referenced in later chapters.

\subsection*{4. Enlightening Examples}
We use realistic dimensions from actual models (BERT, GPT, ViT) and include complete numerical calculations that readers can verify.

\section*{Companion Materials}

A complete set of companion materials is available at \url{https://github.com/[repository-url]}:
\begin{itemize}
    \item Code repository with PyTorch implementations
    \item Jupyter notebooks with interactive visualizations
    \item Exercise solutions for instructors
    \item Slide decks for lectures
\end{itemize}

\vspace{1cm}
\noindent
[Author Names]\\
[Location], 2026
